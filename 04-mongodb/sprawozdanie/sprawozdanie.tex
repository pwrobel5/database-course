\documentclass[a4paper,9pt]{extarticle}	%artykuł

\usepackage[utf8]{inputenc}								%kodowanie znaków
\usepackage{inconsolata}								%Consolas font
\usepackage[T1]{fontenc}								%kodowanie fontu
\usepackage{polski}										%polskie wcięcia itp
\usepackage{graphicx}									%wstawianie grafik
\usepackage{booktabs}									%scalanie kolumn
\usepackage{enumitem}									%lista a), b), c)
\usepackage{multirow}									%scalanie komórek tabeli w pionie
\usepackage{listings}									%wklejanie kodu z SQL
\usepackage{xcolor}										%żeby kolory do kodu działaly
\usepackage[margin=0.5in]{geometry}					%zmiana marginesów
\usepackage[hidelinks]{hyperref}						%linki w spisie treści

\hypersetup{linktoc=all}								%ustawienie linków w spisie treści
\linespread{1.3}										%interlinia 1,5

%\renewcommand\familydefault{\sfdefault}

\lstset{ %
  backgroundcolor=\color{white},   % choose the background color; you must add \usepackage{color} or \usepackage{xcolor}; should come as last argument
  basicstyle=\ttfamily,        % the size of the fonts that are used for the code
  breakatwhitespace=false,         % sets if automatic breaks should only happen at whitespace
  breaklines=true,                 % sets automatic line breaking
  commentstyle=\color{gray},    	% comment style
  extendedchars=true,              % lets you use non-ASCII characters; for 8-bits encodings only, does not work with UTF-8
  frame=single,	                    % adds a frame around the code
  keywordstyle=\color{blue},       % keyword style
  language=Java,             % the language of the code
  morekeywords={*,...},            % if you want to add more keywords to the set
%  numbers=left,                    % where to put the line-numbers; possible values are (none, left, right)
%  numbersep=5pt,                   % how far the line-numbers are from the code
%  numberstyle=\normalsize,			% the style that is used for the line-numbers
  rulecolor=\color{black},         % if not set, the frame-color may be changed on line-breaks within not-black text (e.g. comments (green here))
  showspaces=false,                % show spaces everywhere adding particular underscores; it overrides 'showstringspaces'
  showstringspaces=false,          % underline spaces within strings only
  showtabs=false,                  % show tabs within strings adding particular underscores
  stepnumber=1,                    % the step between two line-numbers. If it's 1, each line will be numbered
  stringstyle=\color{orange},      % string literal style
  tabsize=2	                  		% sets default tabsize to 2 spaces                   
}

\lstset{language=Java,
       }

\lstset{
literate=%
{ą}{{\k{a}}}1
{Ą}{{\k{A}}}1
{ć}{{\'c}}1
{Ć}{{\'{C}}}1
{ę}{{\k{e}}}1
{Ę}{{\k{E}}}1
{ł}{{\l{}}}1
{Ł}{{\L{}}}1
{ń}{{\'n}}1
{Ń}{{\'N}}1
{ó}{{\'o}}1
{Ó}{{\'O}}1
{ś}{{\'s}}1
{Ś}{{\'S}}1
{ż}{{\.z}}1
{Ż}{{\.Z}}1
{ź}{{\'z}}1
{Ź}{{\'Z}}1
}

\begin{document}

\begin{center}

{\LARGE \textbf{Bazy danych \ppauza NoSQL}}


{\Large \textbf{MongoDB \ppauza zadania}}

\end{center}

\begin{flushright}
Autor zadań: Piotr Wróbel

Data laboratorium: 20.11.2019 r.

Data wykonania: 8.12.2019 r.
\end{flushright}

\begin{enumerate}
	\item Wykorzystując bazę danych yelp dataset wykonaj zapytanie i komendy MongoDB, aby uzyskać następujące rezultaty:
	\begin{enumerate}
	  \item Zwróć bez powtórzeń wszystkie nazwy miast w których znajdują się firmy (business).
	  \begin{lstlisting}
db.business.distinct("city");
	  \end{lstlisting}
	  
	  \item Zwróć liczbę wszystkich recenzji, które pojawiły się w roku 2011 i 2012.
	  \begin{lstlisting}
db.review.find({$or: [{date: /2011/}, {date: /2012/}]});
	  \end{lstlisting}
	  
	  \item Zwróć dane wszystkich otwartych (open) firm (business) z pól: id, nazwa, adres.
	  \begin{lstlisting}
db.business.find({open: true}, {business_id:1, name:1, full_address:1});
	  \end{lstlisting}
	  
	  \item Zwróć dane wszystkich użytkowników (user), którzy uzyskali przynajmniej jeden pozytywny głos z jednej z kategorii (funny, useful, cool), wynik posortuj alfabetycznie na podstawie imienia użytkownika.
	  \begin{lstlisting}
// założony indeks w celu przyspieszenia zapytania	  
db.user.ensureIndex({"name": 1});

db.user.find({$or: [
	{'votes.funny': {$gte: 1}}, 
	{'votes.useful': {$gte: 1}}, 
	{'votes.cool': {$gte: 1}}
]}).sort({name: 1});
	  \end{lstlisting}
	  
	  \item Określ, ile każde przedsiębiorstwo otrzymało wskazówek/napiwków (tip) w 2013. Wynik posortuj alfabetycznie na podstawie nazwy firmy.
	  \begin{lstlisting}
// indeks założony w celu j.w.
db.business.ensureIndex({"name": 1});

db.tip.aggregate([
    {"$match": {date: /2013/}},
    {"$group": 
        {_id: "$business_id",
        count: {$sum: 1}        
    }},
    {"$lookup": {
        "from": "business",
        "localField": "_id",
        "foreignField": "business_id",
        "as": "B"
    }},
    {"$project": {_id: 1, count: 1, 'B.name': 1}},
    {"$sort": {'B.name': 1}}
]);
	  \end{lstlisting}
\newpage
	  \item Wyznacz, jaką średnia ocen (stars) uzyskała każda firma (business) na podstawie wszystkich recenzji, wynik posortuj on najwyższego uzyskanego wyniku.
	  \begin{lstlisting}
db.review.aggregate([
    {"$group":
        {_id: "$business_id",
         avg_stars: {$avg: "$stars"}
    }},
    {"$sort": {avg_stars: -1}},
    {"$lookup": {
        "from": "business",
        "localField": "_id",
        "foreignField": "business_id",
        "as": "B"
    }}
]);
	  \end{lstlisting}
	  
	  \item Usuń wszystkie firmy (business), które posiadają ocenę (stars) poniżej 3.
	  \begin{lstlisting}
var cursor = db.review.aggregate([
    {"$group":
        {_id: "$business_id",
         avg_stars: {$avg: "$stars"}
    }},
    {"$match": {avg_stars: {$lt: 3.0}}}
]);

cursor.forEach(function(bus) {
        db.business.remove({business_id: bus._id});
});
	  \end{lstlisting}
	\end{enumerate}
	
	\item Zdefiniuj funkcję umożliwiającą dodanie nowej wskazówki/napiwku (tip). Wykonaj przykładowe wywołanie.
	\begin{lstlisting}
function insertTip(
    user_id, text, business_id, date) {
        db.tip.insert({
            user_id: user_id,
            text: text,
            business_id: business_id,
            likes: 0,
            date: ISODate(date),
            type: "tip"
        });
}

insertTip("5Xh4Qc3rxhAQ_NcNtxLssQ", "bardzo fajne, podoba mi sie","JwUE5GmEO-sH1FuwJgKBlQ", "2018-01-02");
	\end{lstlisting}
	
	\item Zdefiniuj funkcję, która zwróci wszystkie wskazówki/napiwki (tip), w których w tekście znajdzie się fraza podana jako argument. Wykonaj przykładowe wywołanie zdefiniowanej funkcji.
	\begin{lstlisting}
function findTipByPhrase(phrase) {
    db.tip.find({text: {$regex: ".*" + phrase + ".*"}}).forEach(function(res) { print(res); });
}

findTipByPhrase("bardzo fajne");
	\end{lstlisting}
	
	\item Zdefiniuj funkcję, która umożliwi modyfikację nazwy firmy (business) na podstawie id. Id oraz nazwa mają być przekazywane jako parametry.
	\begin{lstlisting}
function modifyBusinessName(business_id, new_name) {
    db.business.update({business_id: business_id}, {$set: {name: new_name}})
}

modifyBusinessName("JwUE5GmEO-sH1FuwJgKBlQ", "Zmiana nazwy");
	\end{lstlisting}
	
	\item Zwróć średnią ilość wszystkich recenzji użytkowników, wykorzystaj map reduce.
	\begin{lstlisting}
var mapF = function() {
    emit(this.user_id, 1);
};

var reduceF = function(user_id, reviewNum) {
    return Array.sum(reviewNum);
};

db.review.mapReduce(mapF, reduceF, {out: "reviews_per_user"});

db.reviews_per_user.aggregate([
    {"$group": 
        {_id: null,
         avg_rev: {$avg: "$value"}}}
]);
	\end{lstlisting}

	\item Odwzoruj wszystkie zadania z punktu 1 w języku programowania (np. JAVA) z pomocą API do MongoDB. Wykorzystaj dla każdego zadania odrębną metodę.
	\begin{enumerate}
	  \item Zwróć bez powtórzeń wszystkie nazwy miast w których znajdują się firmy (business).
	  \begin{lstlisting}
private void distinctBusinessCities() {
	DBCollection businessCollection = db.getCollection("business");
	for(Object o: businessCollection.distinct("city")) {
		System.out.println(o);
}
}
	  \end{lstlisting}
	  
	  \item Zwróć liczbę wszystkich recenzji, które pojawiły się w roku 2011 i 2012.
	  \begin{lstlisting}
private void reviewsBetweenYears(int yearFrom, int yearTo) {
	DBCollection reviewCollection = db.getCollection("review");

	Pattern yearFromPattern = Pattern.compile(String.valueOf(yearFrom), Pattern.CASE_INSENSITIVE);
	Pattern yearToPattern = Pattern.compile(String.valueOf(yearTo), Pattern.CASE_INSENSITIVE);

	ArrayList<BasicDBObject> orArgs = new ArrayList<>();
	orArgs.add(new BasicDBObject("date", yearFromPattern));
	orArgs.add(new BasicDBObject("date", yearToPattern));
	BasicDBObject query = new BasicDBObject("$or", orArgs);

	DBCursor result = reviewCollection.find(query);

	try {
		while(result.hasNext()) {
			System.out.println(result.next());
		}
	} finally {
		result.close();
	}
}
	  \end{lstlisting}
	  
	  \item Zwróć dane wszystkich otwartych (open) firm (business) z pól: id, nazwa, adres.
	  \begin{lstlisting}
private void openedBusiness() {
	DBCollection businessConnection = db.getCollection("business");

	BasicDBObject query = new BasicDBObject("open", true);

	DBObject projectionFields = new BasicDBObject("business_id", 1);
	projectionFields.put("name", 1);
	projectionFields.put("full_address", 1);

	DBCursor result = businessConnection.find(query, projectionFields);

	try {
		while(result.hasNext()) {
			System.out.println(result.next());
		}
	} finally {
		result.close();
	}
}
	  \end{lstlisting}
	  
	  \item Zwróć dane wszystkich użytkowników (user), którzy uzyskali przynajmniej jeden pozytywny głos z jednej z kategorii (funny, useful, cool), wynik posortuj alfabetycznie na podstawie imienia użytkownika.
	  \begin{lstlisting}
private void usersWithPositiveVotes() {
	DBCollection userCollection = db.getCollection("user");

	userCollection.createIndex(new BasicDBObject("name", 1));

	ArrayList<BasicDBObject> orArgs = new ArrayList<>();
	orArgs.add(new BasicDBObject("votes.funny", new BasicDBObject("$gte", 1)));
	orArgs.add(new BasicDBObject("votes.useful", new BasicDBObject("$gte", 1)));
	orArgs.add(new BasicDBObject("votes.cool", new BasicDBObject("$gte", 1)));

	BasicDBObject query = new BasicDBObject("$or", orArgs);
	BasicDBObject sortingOrder = new BasicDBObject("name", 1);

	DBCursor result = userCollection.find(query).sort(sortingOrder);

	try {
		while(result.hasNext()) {
			System.out.println(result.next());
		}
	} finally {
		result.close();
	}
}
	  \end{lstlisting}
	  
	  \item Określ, ile każde przedsiębiorstwo otrzymało wskazówek/napiwków (tip) w 2013. Wynik posortuj alfabetycznie na podstawie nazwy firmy.
	  \begin{lstlisting}
private void businessTipsNumber(int year) {
	DBCollection businessCollection = db.getCollection("business");
	DBCollection tipCollection = db.getCollection("tip");

	businessCollection.createIndex(new BasicDBObject("name", 1));

	Pattern yearPattern = Pattern.compile(String.valueOf(year), Pattern.CASE_INSENSITIVE);

	DBObject match = new BasicDBObject("$match", new BasicDBObject("date", yearPattern));

	DBObject groupFields = new BasicDBObject("_id", "$business_id");
	groupFields.put("count", new BasicDBObject("$sum", 1));
	DBObject group = new BasicDBObject("$group", groupFields);

	DBObject lookupFields = new BasicDBObject("from", "business");
	lookupFields.put("localField", "_id");
	lookupFields.put("foreignField", "business_id");
	lookupFields.put("as", "B");
	DBObject lookup = new BasicDBObject("$lookup", lookupFields);

	DBObject projectFields = new BasicDBObject("_id", 1);
	projectFields.put("count", 1);
	projectFields.put("B.name", 1);
	DBObject project = new BasicDBObject("$project", projectFields);

	DBObject sort = new BasicDBObject("$sort", new BasicDBObject("B.name", 1));

	List<DBObject> pipeline = Arrays.asList(match, group, lookup, project, sort);
	AggregationOutput output = tipCollection.aggregate(pipeline);

	for(DBObject result: output.results()) {
		System.out.println(result);
	}
}
	  \end{lstlisting}
	  
	  \item Wyznacz, jaką średnia ocen (stars) uzyskała każda firma (business) na podstawie wszystkich recenzji, wynik posortuj on najwyższego uzyskanego wyniku.
	  \begin{lstlisting}
private void businessStars() {
	DBCollection reviewCollection = db.getCollection("review");

	DBObject groupFields = new BasicDBObject("_id", "$business_id");
	groupFields.put("avg_stars", new BasicDBObject("$avg", "$stars"));
	DBObject group = new BasicDBObject("$group", groupFields);

	DBObject sort = new BasicDBObject("$sort", new BasicDBObject("avg_stars", -1));

	DBObject lookupFields = new BasicDBObject("from", "business");
	lookupFields.put("localField", "_id");
	lookupFields.put("foreignField", "business_id");
	lookupFields.put("as", "B");
	DBObject lookup = new BasicDBObject("$lookup", lookupFields);

	List<DBObject> pipeline = Arrays.asList(group, sort, lookup);
	AggregationOutput output = reviewCollection.aggregate(pipeline);

	for(DBObject result: output.results()) {
		System.out.println(result);
	}
}
	  \end{lstlisting}
	  
	  \item Usuń wszystkie firmy (business), które posiadają ocenę (stars) poniżej 3.
	  \begin{lstlisting}
private void deleteBusinessUnderThreshold(double threshold) {
	DBCollection reviewCollection = db.getCollection("review");

	DBObject groupFields = new BasicDBObject("_id", "$business_id");
	groupFields.put("avg_stars", new BasicDBObject("$avg", "$stars"));
	DBObject group = new BasicDBObject("$group", groupFields);

	DBObject match = new BasicDBObject("$match", new BasicDBObject("avg_stars", new BasicDBObject("$lt", threshold)));

	List<DBObject> pipeline = Arrays.asList(group, match);
	AggregationOutput output = reviewCollection.aggregate(pipeline);
	DBCollection businessCollection = db.getCollection("business");

	for(DBObject result: output.results()) {
		businessCollection.remove(new BasicDBObject("business_id", result.get("_id")));
	}
}
	  \end{lstlisting}
	\end{enumerate}
	
	\item Zaproponuj bazę danych składającą się z 3 kolekcji pozwalającą przechowywać dane dotyczące: studentów, przedmiotów oraz sal zajęciowych. W bazie wykorzystaj: pola proste, złożone i tablice. Zaprezentuj strukturę dokumentów w formie JSON dla przykładowych danych.
	\begin{enumerate}
	  \item studenci \ppauza każdy student poza podstawowymi danymi osobowymi (imię, nazwisko, adres, pesel) ma także pola z~danymi uczelnianymi (numer legitymacji, wydział, kierunek i~rodzaj studiów, bieżący rok oraz listę kursów na które jest zapisany, przykładowe dane:
	  \begin{lstlisting}
student1 = {
    _id: 1,
    first_name: "Michał",
    last_name: "Kowalski",
    pesel: "94091801234",
    enrolled_courses: [
        {course_id: 11, type: "obligatory"},
        {course_id: 12, type: "facultative"}
    ],
    student_card_ID: 1100119,
    address: {
	street: "Mickiewicza 29/1",
	postal_code: "30-856",
	city: "Kraków",
	country: "Poland"
    },
    faculty: "WIET",
    field_of_study: "informatyka",
    study_type: {
	mode: "dzienne",
	step: "inżynierskie"
    }
    year: "III"
}
	  \end{lstlisting}
	  \item przedmioty \ppauza każdy przedmiot ma swoją nazwę, liczbę punktów ECTS, oznaczenie czy kończy się egzaminem, listę terminów i~miejsc zajęć, wraz z~oznaczeniem co ile tygodni się odbywają, liczbę godzin poszczególnych składowych kursu oraz listę kursów, które należy mieć zaliczone przed przystąpieniem do danego, przykładowe dane:
	  \begin{lstlisting}
course11 = {
    _id: 11,
    name: "Bazy danych",
    ects: 3,
    has_exam: false,
    classes_terms: [
        {room_id: 21,
         from: "16:15",
         to: "17:45",
         day: "Wednesday",
	 type: "laboratory",
         week_interval: 2
        },
        {room_id: 22,
         from: "12:50",
         to: "14:20",
         day: "Thursday",
	 type: "lecture"
         week_interval: 2
        }
    ],
    number_of_hours: {
        lecture: 15,
        laboratory: 15
    }
    prerequisites: [
        {course_id: 10},
        {course_id: 15}
    ]
}
	  \end{lstlisting}
	  \item sale zajęciowe \ppauza każda sala ma swój numer, oznaczenie budynku, w~którym się znajduje, liczbę dostępnych miejsc oraz swój typ, przykładowe dane;
	  \begin{lstlisting}
room21 = {
    _id: 21,
    name: "4.38",
    building: "D17",
    number_of_places: 15,
    room_type: "computer laboratory"
}
	  \end{lstlisting}
	\end{enumerate}
\end{enumerate}
\end{document}

